\section{Кинематика вращательного движения материальной точки}
%Сив42
\begin{wrapfigure}[8]{r}{4cm}
\includegraphics[width=0.3\textwidth]{maxAccel.png}
\caption{}
\label{maxAccel}
\end{wrapfigure}
\AddProb Точка движется равномерно по плоской траектории, изображенной на рис. \ref{maxAccel}. В каком месте траектории ускорение точки будет максимальным?
%Сив43
\AddProb Луна обращается вокруг Земли с периодом $T = 27,3$ сут относительно звезд. Средний радиус орбиты Луны $R = 3,8 \cdot 10^5$ км. Найти линейную скорость $v$ движения Луны вокруг Земли и ее нормальное ускорение $a$. 
%Сив44
\AddProb Каковы будут графики зависимости от времени абсолютных величин скорости и ускорения при равномерном движении точки по кругу?
%Сив54
\AddProb Якорь электромотора, вращавшегося с частотой $N$ оборотов в секунду, двигаясь после выключения тока равнозамедленно, остановился, сделав $n$ оборотов. Найти угловое ускорение якоря после выключения тока.
%Сив58
\AddProb Автомобиль, движущийся со скоростью 40 км/ч, проходит закругление шоссе с радиусом кривизны 200 м. На повороте шофер тормозит машину, сообщая ей ускорение 0,3 м/с\textsuperscript{2}. Найти нормальное и полное ускорение автомобиля на повороте. Как направлен вектор полного ускорения $\vec{a}$ по отношению к радиусу кривизны $R$ закругления шоссе?
\begin{wrapfigure}[9]{r}{4cm}
\includegraphics[width=0.3\textwidth]{cycloid.png}
\caption{}
\label{cycloid}
\end{wrapfigure}
%Сив62
\AddProb Колесо радиуса $R$ равномерно катится без скольжения по горизонтальному пути со скоростью $v$. Найти координаты $x$ и $y$ произвольной точки $A$ на ободе колеса, выразив их как функции времени $t$ или угла поворота колеса $\varphi$, полагая, что при $t = 0$ $\varphi = 0$, $x = 0$, $y = 0$ (рис. \ref{cycloid}). По найденным выражениям для $x$ и $y$ построить график траектории точки на ободе колеса.
%Сив45
\AddProb Найти среднюю угловую скорость искусственного спутника Земли, если период обращения его по орбите вокруг Земли составляет 105 мин.
%Сив49
\AddProb Найти линейную скорость Земли, вызванную ее орбитальным движением. Средний радиус земной орбиты равен $R = 1,5 \cdot 10^8$ км.
%Сив55
\AddProb Автомобиль движется со скоростью 60 км/ч. Сколько оборотов в секунду делают его колеса, если они катятся по шоссе без скольжения, а внешний диаметр покрышек колес равен 60 см.
%Сив57
\AddProb Разматывая веревку и вращая без скольжения вал ворота, ведро опускается в колодец с ускорением 1 м/с\textsuperscript{2}. С каким угловым ускорением вращается вал ворота? Как зависит от времени угол поворота вала? Радиус вала ворота равен 25 см.
%Сив66
\AddProb Представление о величине и направлении вектора полного ускорения при ускоренном вращательном движении (например, для точек якоря электромотора при его пуске) можно получить, рассмотрев следующую задачу. Точка движется по окружности радиусом $R$ с постоянным тангенциальным ускорением $a_{\tau}$, но без начальной скорости. Найти нормальное и полное ускорения точки, выразив их: 1) как функцию от времени $t$ и ускорения $a_{\tau}$; 2) как функцию от углового ускорения $\varepsilon$ и угла поворота $\varphi$ радиуса-вектора точки из его начального положения. Найти угол $\beta$ между направлением вектора полного ускорения точки и ее
радиусом-вектором.

%Сив59
\begin{wrapfigure}[10]{r}{4cm}
\includegraphics[width=0.3\textwidth]{rollingWheel.png}
\caption{}
\label{rollingWheel}
\end{wrapfigure}
\AddProb Колесо радиуса $R$ катится без скольжения по горизонтальной дороге со скоростью $v_0$ (рис. \ref{rollingWheel}). Найти горизонтальную компоненту $v_x$ линейной скорости движения произвольной точки на ободе колеса, вертикальную компоненту $v_y$ этой скорости и модуль полной скорости для этой же точки. Найти значение угла $\alpha$ между вектором полной скорости точек на ободе колеса и направлением поступательного движения его оси. Показать, что направление
вектора полной скорости произвольной точки $A$ на ободе колеса всегда перпендикулярно к прямой $AB$ и проходит через высшую точку катящегося колеса. Показать, что для точки $A$ $v = BA \omega$. Построить график распределения скоростей для всех точек на вертикальном диаметре (в данный момент времени) катящегося без скольжения колеса. Выразить все искомые величины через $v_0$, $R$ и угол $\varphi$, составленный верхним вертикальным радиусом колеса и радиусом, проведенным из
центра колеса $O$ в исследуемую точку его обода $A$.
%Сив64
\AddProb Автомобиль с колесами радиуса $R$ движется со скоростью $v$ по горизонтальной дороге, причем $v^2 > Rg$, где $g$ -- ускорение свободного падения. На какую максимальную высоту $H_{max}$ может быть заброшена вверх грязь, срывающаяся с колес автомобиля? Указать положение той
точки на покрышке колеса, с которой при данной скорости движения автомобиля грязь будет забрасываться выше всего. Сопротивление воздуха движению отброшенной вверх грязи не учитывать.

%Иродов1.57
\begin{wrapfigure}[10]{r}{4cm}
\includegraphics[width=0.3\textwidth]{rollingCylinder.png}
\caption{}
\label{rollingCylinder}
\end{wrapfigure}
\AddProb Круглый конус с углом полураствора $\alpha = 30^{\circ}$ и радиусом основания $R = 5,0$ см катится равномерно без скольжения по горизонтальной плоскости, как показано на рис. \ref{rollingCylinder}. Вершина конуса закреплена шарнирно в точке $O$, которая находится на одном уровне с точкой $C$ -- центром основания конуса. Скорость точки $C$ равна $v = 10,0$ см/с. Найти модули: 1) угловой скорости конуса; 2) углового ускорения конуса.

\clearpage