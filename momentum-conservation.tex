\section{Закон сохранения импульса}
%Сив187
\AddProb С какой скоростью $v$ после горизонтального выстрела из винтовки стал двигаться стрелок, стоящий на весьма гладком льду? Масса стрелка с винтовкой и снаряжением составляет 70 кг, а масса пули 10 г и ее начальная скорость 700 м/с.
%Сив188
\AddProb Определить силу, с которой винтовка действует на плечо
стрелка при выстреле, если считать, что со стороны винтовки действует постоянная сила и смещает плечо стрелка на $S = 1,5$ см, а пуля покидает ствол мгновенно. Масса винтовки 5 кг, масса пули 10 г, и скорость ее при вылете равна $v = 500$ м/с.
%Сив191
\AddProb Три лодки одинаковой массы $m$ идут в кильватер (друг за
другом) с одинаковой скоростью $v$. Из средней лодки одновременно
в переднюю и заднюю лодки бросают со скоростью $u$ относительно
лодки грузы массы $m_1$. Каковы будут скорости лодок после переброски грузов?
%Сив194
\AddProb В шар массы $m_1$, движущийся со скоростью $v_1$, ударяется
другой шар массы $m_2$, догоняющий первый в том же направлении со
скоростью $v_2$. Считая удар абсолютно неупругим, найти скорости шаров после удара и их кинетическую энергию.
%Сив190
\AddProb Снаряд разрывается в верхней точке траектории на высоте $h = 19,6$ м на две одинаковые части. Через секунду после взрыва одна
часть падает на Землю под тем местом, где произошел взрыв. На
каком расстоянии $S_2$ от места выстрела упадет вторая часть снаряда, если первая упала на расстоянии $S_1 = 1000$ м от места выстрела? Сил сопротивления воздуха при решении задачи не учитывать.
%Сив193
\AddProb Две лодки идут навстречу параллельным курсом. Когда лодки
находятся друг против друга, с каждой лодки во встречную перебрасывается мешок массой в 50 кг, в результате чего первая лодка 
останавливается, а вторая идет со скоростью 8,5 м/с в прежнем направлении. Каковы были скорости лодок до обмена мешками, если массы лодок с грузом равны 500 кг и 1 т соответственно?
%Сив199
\AddProb С какой скоростью $v$ должен лететь снаряд массы $m$ = 10 кг, чтобы при ударе о судно массы $М$ = 100 т последнее получило скорость $v_1$ = 0,1 м/с? Удар считать неупругим.
%Сив200
\AddProb Ледокол, ударяясь о льдину массы $М$, отбрасывает ее, сообщив ей скорость $v$ [м/с]. Положим, что давление ледокола на льдину нарастает равномерно во времени при сближении ледокола со льдиной и также равномерно убывает, когда они расходятся. Найти при этих условиях максимальную силу давления льдины на борт корабля, если удар продолжался $\tau$ [с].
%Сив201
\AddProb В одном изобретении предлагается на ходу наполнять платформы поезда углем, падающим вертикально на платформу из соответствующим образом устроенного бункера. Какова должна быть приложенная к платформе сила тяги, если на нее погружают 10 т угля
за 2 с, и за это время она проходит равномерно 10 м? Трением при
движении платформы можно пренебречь.
%Сив233
\AddProb Лодка массы $M$ с находящимся в ней человеком массы $m$ неподвижно стоит на спокойной воде. Человек начинает идти вдоль
по лодке со скоростью $\vec{u}$ и относительно лодки. С какой скоростью $\vec{w}$ будет двигаться человек относительно воды? С какой скоростью $\vec{v}$ будет при этом двигаться лодка относительно воды? Сопротивление воды движению лодки не учитывать.
%Сив234
\AddProb Пусть человек прошел вдоль по лодке, описанной в преды-
предыдущей задаче, его перемещение составило $\vec{l}$. Каковы будут при этом смещения лодки $\vec{S_1}$ и человека $\vec{S_2}$ относительно воды?
%Иродов1.129
\AddProb В момент, когда скорость падающего тела составила $v_0 = 4,О$ м/с, оно разорвалось на три одинаковых осколка. Два осколка разлетелись в горизонтальной плоскости под прямым углом друг к другу со скоростью $v = 5,0$ м/с каждый. Найти скорость третьего осколка сразу после разрыва.
%Иродов1.132
\AddProb Цепочка массы $m = 1,00$ кг и длины $l = 1,40$ м висит на нити, касаясь поверхности стола своим нижним концом. После пережигания нити цепочка упала на стол. Найти полный импульс, который она передала столу.
%МФТИ3.5
\AddProb С какой силой пожарный должен удерживать пожарный шланг, вода из которого выбрасывается в виде потока мелких капель в количестве $Q = 10$ кг/с и может добивать до высоты $h = 10$ м? Оцените, какая сила будет действовать на стену, если на нее направить струю из этого шланга под углом $45^{\circ}$ (оценку провести в двух предельных случаях -- для упругого и для неупругого соударения капель со стенкой).
\clearpage