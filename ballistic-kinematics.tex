\section{Движение тела, брошенного под углом к горизонту}
%Сив21
\AddProb	 Из артиллерийского орудия произведен выстрел под углом $\varphi$ к горизонту. Величина начальной скорости снаряда $v_0$. Исследовать аналитически движение снаряда, пренебрегая сопротивлением воздуха полету снаряда и кривизной поверхности Земли. Найденные зависимости изобразить графически. Найти: 1) вертикальную и горизонтальную компоненты вектора скорости $\vec{v}$ и абсолютную величину скорости как функцию времени; 2) время $Т$ полета снаряда от орудия до падения на землю; 3) зависимость от времени угла $\alpha$ между вектором скорости снаряда и горизонтом; 4) декартовы координаты (ось $X$ — горизонтальное направление, ось $Y$ — вертикальное направление) снаряда как функции времени; 5) уравнение траектории снаряда $y = f(x)$ (построить согласно этому уравнению траекторию полета снаряда); 6) максимальную высоту $H_{max}$ полета снаряда над землей; 7) горизонтальную дальность $L_{max}$ полета снаряда как функцию его начальной скорости и угла возвышения орудия; 8) При каком угле возвышения $\varphi^*$ дальность будет максимальной при заданной начальной скорости снаряда?
%Сив30
\AddProb Самолет летит на высоте $h$ горизонтально по прямой со скоростью $v$. Летчик должен сбросить бомбу в цель, лежащую впереди самолета. Под каким углом $\alpha$ к вертикали он должен видеть цель в момент выпуска бомбы? Каково в этот момент расстояние $l$ от цели до точки, над которой находится самолет? Сопротивление воздуха движению бомбы не учитывать.
%Сив32
\AddProb Цель, находящаяся на холме, видна с места расположения орудия под углом $\alpha$ к горизонту. Дистанция (расстояние по горизонтали от орудия до цели) равна $L$. Стрельба по цели производится при угле возвышения $\beta$. Определить начальную скорость $v_0$ снаряда, попадающего в цель. Сопротивление воздуха не учитывать.
%Сив18
\AddProb Какой начальной скоростью $v_0$ должна обладать сигнальная ракета, выпущенная из ракетницы под углом $45^{\circ}$ к горизонту, чтобы она вспыхнула в наивысшей точке своей траектории, если время горения запала ракеты 6 с? Сопротивление воздуха движению ракеты не учитывать.
%ПодОл3-1
\AddProb Точка движется согласно уравнениям: $x = 2t + 6, y = t^2$. Проходит ли ее траектория через точку $x = 10, y = 10$? Найдите величину и направление ускорения точки?
%Сив23
\AddProb Из трех труб, расположенных на земле, с одинаковой скоростью бьют струи воды: под углом в $60^{\circ}$, $45^{\circ}$ и $30^{\circ}$ к горизонту. Найти отношение наибольших высот $H$ подъема струй воды, вытекающих из каждой трубы, и отношение дальностей падения $L$ воды на землю. Сопротивление воздуха движению водяных струй не учитывать.
%ПодОл3-5
\AddProb Под каким углом к горизонту надо бросить камень, чтобы дальность его полета была втрое больше максимальной высоты подъема?
%Сив24
\AddProb На какое максимальное расстояние $L$ можно бросить мяч в спортивном зале высотой 8 м, если мяч имеет начальную скорость 20 м/с? Какой угол $\varphi$ с полом зала должен в этом случае составлять вектор начальной скорости мяча? Считать, что высота начальной точки траектории мяча над полом мала по сравнению с высотой зала. Мяч во время полета не должен ударяться о потолок зала. Сопротивлением воздуха полету мяча пренебречь.
%ПодОл3-8 
\AddProb Камень бросили с крутого берега реки вверх под углом $30^{\circ}$ к горизонту со скоростью $v_0 = 10$ м/с. С какой скоростью он упал в воду, если время полета $t = 2$ с?
%Иродов1.10
\AddProbДва тела бросили одновременно из одной точки: одно — вертикально вверх, другое - под углом $\alpha = 60^{\circ}$ к горизонту. Начальная скорость каждого тела $v_0 = 25$ м/с. Найти расстояние между телами через $t = 1,70$ с.
%Иродов1.11
\AddProb Два шарика бросили одновременно из одной точки в горизонтальном направлении в противоположные стороны со скоростями $v_1 = 3,0$ м/с и $v_2 = 4,0$ м/с. Найти расстояние между шариками в момент, когда их скорости окажутся взаимно перпендикулярными.

\begin{wrapfigure}[7]{r}{4cm}
\includegraphics[width=0.3\textwidth]{duckHunter.png}
\caption{}
\label{duckHunter}
\end{wrapfigure}
%ПодОл3-13
\AddProb Утка летела по горизонтальной прямой с постоянной скоростью $u$. В нее бросил камень неопытный охотник, причем бросок был сделан без упреждения, т.е. в момент броска скорость камня $v$ была направлена как раз на утку под углом $\alpha$ к горизонту (рис. \ref{duckHunter}). На какой высоте летела утка, если камень все же попал в нее? С какой стороны камень ударил утку?
%Сив20
\AddProb Шарик, которому сообщена горизонтальная скорость $v$, падает на горизонтальную плиту с высоты $h$. При каждом ударе о плиту вертикальная составляющая скорости уменьшается (отношение вертикальной составляющей скорости после удара к ее значению до удара
постоянно и равно $\alpha$). Определить, на каком расстоянии $х$ от места бросания отскоки шарика прекратятся. Считать, что трение отсутствует, так что горизонтальная составляющая скорости шарика $v$ не меняется.
\clearpage