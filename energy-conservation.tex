\section{Работа. Закон сохранения энергии}
%Сив162
\AddProb Коэффициент трения между некоторым телом и плоскостью,
наклоненной под углом $45^{\circ}$ к горизонту, равен 0,2. На какую высоту поднимается это тело, скользя по наклонной плоскости, если ему будет сообщена скорость 10 м/с, направленная вверх вдоль плоскости? Какова будет скорость тела, когда оно вернется в нижнюю исходную точку своего движения?
%Сив166
\AddProb Из залитого подвала, площадь пола которого равна 50 м\textsuperscript{2}, требуется выкачать воду на мостовую. Глубина воды в подвале 1,5 м, а расстояние от уровня воды в подвале до мостовой 5 м. Найти работу, которую необходимо затратить для откачки воды.
%Сив170
\AddProb Определить среднюю полезную мощность при выстреле из гладкоствольного ружья, если известно, что пуля массы $m$ вылетает из ствола со скоростью $v_0$, а длина канала ствола $l$ (давление пороховых газов считать постоянным во все время нахождения снаряда в канале ствола).

%Сив196
\begin{wrapfigure}[6]{r}{4.8cm}
\includegraphics[width=0.4\textwidth]{2BallsSpring.png}
\caption{}
\label{2BallsSpring}
\end{wrapfigure}
\AddProb На гладком горизонтальном столе лежит шар массы $m_1$, соединенный с пружиной жесткости $k$. Второй конец пружины закреплен
(рис. \ref{2BallsSpring}). Происходит лобовое упругое соударение этого шара с другим шаром, масса которого $m_2$ меньше $m_1$, а скорость равна $v$. В какую сторону будет двигаться второй шар после удара? Определить амплитуду колебаний первого шара после соударения.

\begin{wrapfigure}[10]{r}{3.5cm}
\includegraphics[width=0.3\textwidth]{ballisticPendulum.png}
\caption{}
\label{ballisticPendulum}
\end{wrapfigure}
%Сив227
\AddProb Баллистический маятник — это маятник, употребляющийся для определения скорости снаряда. Принцип его действия заключается в том, что снаряд, скорость которого следует измерить, ударяется в тело маятника длины $l$ (рис. \ref{ballisticPendulum}). Если известны условия удара и массы снаряда $m$ и маятника $M$, то по углу отклонения маятника $\alpha$ можно вычислить скорость $v$ снаряда до удара. Показать, как это сделать для случая, когда снаряд после удара застревает в маятнике.
%Сив163
\AddProb Какую работу надо совершить, чтобы втащить (волоком) тело
массы $m$ на горку с длиной основания $L$ и высотой $H$, если коэффициент трения между телом и поверхностью горки равен $\mu$? Как изменится ответ, если угол наклона поверхности горки с горизонтом может меняться вдоль горки, но его знак остается постоянным.
%Сив168
\AddProb Оконная штора массой в 1 кг и длиной 2 м свертывается на тонкий валик наверху окна. Какая при этом совершается работа? Трением пренебречь.
%Сив176
\AddProb Какую мощность $N$ затрачивает лошадь на движение саней,
если она тянет их в гору равномерно со скоростью $v$? Масса саней $m$ и трение между санями и поверхностью горы постоянно, коэффициент
трения $\mu$. Угол наклона горы $\alpha$.
%МФТИ4.8
\AddProb Шайба массы $m$, скользя по льду, сталкивается с неподвижной шайбой массы $3m$. Считая удар упругим и центральным, определить, на какое расстояние $S$ разлетятся шайбы, если скорость первой шайбы перед ударом была $v$, а коэффициент трения между шайбами и льдом равен $\mu$.
%МФТИ4.2
\AddProb Математический маятник длиной $l$ находится в положении равновесия. Определите, какую скорость $u$ надо сообщить грузу, чтобы он мог совершить полный оборот, для двух случаев: груз подвешен а) на жестком стержне и б) на нерастяжимой нити.

\begin{wrapfigure}[11]{r}{5cm}
\includegraphics[width=0.4\textwidth]{2Springs.png}
\caption{}
\label{2Springs}
\end{wrapfigure}
%Сив179
\AddProb Определить отношение потенциальных энергий деформации $U_1$ и $U_2$ двух пружин с коэффициентами упругости $k_1$ и $k_2$ в двух
случаях: 1) пружины соединены последовательно и растягиваются грузом $Р$ (рис. \ref{2Springs}а); 2) пружины висят параллельно, причем груз $Р$ подвешен в такой точке, что обе пружины растягиваются на одну и ту же величину (рис. \ref{2Springs}б). Деформацией пружин под действием собственного веса пренебречь.

\begin{wrapfigure}[4]{r}{5cm}
\includegraphics[width=0.4\textwidth]{3BallsSpring.png}
\caption{}
\label{3BallsSpring}
\end{wrapfigure}
%Сив197
\AddProb Система состоит из двух шариков с массами $m$ и $M$, соединенных между собой невесомой пружиной с коэффициентом жесткости $k$ (рис. \ref{3BallsSpring}). Третий шарик с массой $m$, движущийся вдоль оси пружины со скоростью $v$, претерпевает упругое столкновение с шариком $m$, как указано на рис. \ref{3BallsSpring}. Считая шарики абсолютно жесткими, найти после столкновения: 1) кинетическую энергию $K$ движения системы как целого; 2) внутреннюю энергию системы $E_1$; 3) амплитуду колебаний одного шарика относительно другого $A$. До удара система покоилась, а пружина не
была деформирована. Какие шарики могут рассматриваться как абсолютно жесткие?	
%Сив232
\AddProb От поезда, идущего с постоянной скоростью, отрывается последний вагон, который проходит путь $l$ и останавливается. На каком расстоянии от вагона в момент его остановки будет находиться поезд, если тяга паровоза постоянна, а трение каждой части поезда не зависит от скорости и пропорционально ее весу? Масса поезда до момента отрыва вагона $M$, масса вагона $m$.
%Сив229
\AddProb Два маятника в виде шариков разных масс $m_1$ и $m_2$ свободно подвешены на нитях разной длины $l_1$ и $l_2$ так, что шарики соприкасаются. Первый маятник отводят в плоскости нитей на угол $\alpha$ от первоначального положения и отпускают. Происходит центральный удар шариков. На какие углы $\alpha_1$ и $\alpha_2$ относительно отвесной линии отклонятся маятники после удара (углы считать малыми, удар считать упругим)?
%Чертов2.69
\AddProb Камешек скользит с наивысшей точки купола, имеющей форму полусферы. Какую дугу $\alpha$ опишет камушек, прежде чем оторвется от поверхности купола? Трением пренебречь.
\clearpage