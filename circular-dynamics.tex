\section{Динамика движения материальной точки по окружности}

\introProblems

\begin{ex} %Сив260
С какой начальной скоростью $v_0$ должен вылететь снаряд из орудия в горизонтальном направлении, чтобы двигаться вокруг Земли, не падая на нее? Каким ускорением будет обладать снаряд при этом (Радиус Земли $R = 6,4 \cdot 10^3$ км).
\begin{ans}
$v_0 = \sqrt{gR} \approx 8$ км/с.
\end{ans}
\end{ex}	

\begin{ex} %Сив273
Самолет делает «мертвую петлю» радиуса $R = 100$ м и движется по ней со скоростью $v = 280$ км/ч. С какой силой тело летчика массой в 80 кг будет давить на сиденье самолета в верхней и нижней точках петли?
\begin{ans}
$F_1 \approx 5,5$ кН; $F_2 \approx 4$ кН.
\end{ans}
\end{ex}	

\begin{ex}  %Сив264
На внутренней поверхности конической воронки с углом $2\alpha$ при вершине (рис. \ref{Conus}) на высоте $h$ от вершины находится малое тело. Коэффициент трения между телом и поверхностью воронки равен $\mu$. Найти минимальную угловую скорость вращения конуса вокруг вертикальной оси, при которой тело будет неподвижно в воронке.

\begin{figure}[h]
\centering
\includegraphics[width=0.3\textwidth]{Conus.png}
\caption{}
\label{Conus}
\end{figure}

\begin{ans}
$\omega^2 = \frac{g(\cos \alpha - \mu \sin \alpha)}{h \tg \alpha (\sin \alpha + \mu \cos \alpha)}$.
\end{ans}
\end{ex}	

\begin{ex} %Сив265
Велосипедист при повороте по кругу радиуса $R$ наклоняется внутрь закругления так, что угол между плоскостью велосипеда и землей равен $\alpha$. Найти скорость $v$ велосипедиста.
\begin{ans}
$v = \sqrt{Rg \ctg \alpha}$.
\end{ans}
\end{ex}	

\begin{ex} %Сив285
Металлическое кольцо, подвешенное на нити к оси центробежной машины, как указано на рис. \ref{RotationRing}, равномерно вращается с угловой скоростью $\omega$. Нить составляет угол $\alpha$ с осью. Найти расстояние от центра кольца до оси вращения.

\begin{figure}
\centering
\includegraphics[width=0.25\textwidth]{RotationRing.png}
\caption{}
\label{RotationRing}
\end{figure}

\begin{ans}
$x = (g/\omega^2) \tg \alpha$.
\end{ans}
\end{ex}	

\qualProblems

\begin{ex}
На нити подвешен шарик. Нить приводят в горизонтальное положение и отпускают шарик. В каких точках траектории его ускорение направлено: а) вертикально вверх; б) вертикально вниз?
\end{ex}	

\begin{ex}
В кабине лифта на нити, подвешенной к потолку, качается небольшой грузик. Как будет двигаться грузик относительно лифта, если в какой-то момент лифт начнет свободно падать?
\end{ex}	

\begin{ex}
На правые или левые рессоры оседает автомобиль при левом повороте?
\end{ex}	

\begin{ex}
На круглой горизонтальной платформе находится небольшое тело. Куда направлена сила трения, действующая на это тело при раскручивании платформы вокруг вертикальной оси до его соскальзывания?
\end{ex}	

\begin{ex}
Как положить находящийся на столе шарик в банку, не прикасаясь к нему руками и не подкатывая его к краю стола?
\end{ex}	

\begin{ex}
На центробежной машине укреплен диск от серены, на него поставлена зажженная свеча и закрыта стеклянной колбой (рис. \ref{centrofugFire}). При вращении диска пламя отклоняется к оси вращения. Почему?
\end{ex}	

\begin{figure}
\centering
\includegraphics[width=0.4\textwidth]{centrofugFire.png}
\caption{}
\label{centrofugFire}
\end{figure}

\begin{ex} %Сив269
Шофер, едущий на автомобиле по горизонтальной площади в тумане, внезапно заметил недалеко впереди себя стену, перпендикулярную к направлению движения. Что выгоднее: затормозить или повернуть в сторону, чтобы предотвратить аварию?
\begin{ans}
Затормозить.
\end{ans}
\end{ex}	

\simpleProblems

\begin{ex} %Сив257
Найти силу $F$, с которой тележка массы $m$, движущаяся со скоростью $v$, давит на мост в одном из следующих случаев: 1) горизонтальный мост; 2) выпуклый мост (рис. \ref{Bridge}); 3) вогнутый мост. (Для случаев 2) и 3) силу $F$ определить для наивысшей и наинизшей точек моста).
\begin{ans}
1) $F=mg$; 2) $F=mg-\frac{mv^2}{R}$; 3) $F=mg+\frac{mv^2}{R}$.
\end{ans}
\end{ex}	

\begin{figure}[h]
\centering
\includegraphics[width=0.5\textwidth]{Bridge.png}
\caption{}
\label{Bridge}
\end{figure}

\begin{ex} %Сив258
Тело движется прямолинейно с постоянной скоростью $v$ по горизонтальной поверхности стола, которая имеет закругленный край с постоянным радиусом закругления, равным $R$ (рис. \ref{Rounding}). Каково должно быть минимальное значение скорости $v_0$, чтобы тело, падая со стола, не касалось закругления?
\begin{ans}
$v_0 = \sqrt{gR}$.
\end{ans}
\end{ex}	

\begin{figure}[h]
\centering
\includegraphics[width=0.5\textwidth]{Rounding.png}
\caption{}
\label{Rounding}
\end{figure}

\begin{ex}  %Сив263
Каков должен быть минимальный коэффициент трения скольжения к между шинами автомобиля и асфальтом, чтобы автомобиль мог пройти закругление с радиусом $R = 200$ м при скорости $v = 100$ км/ч?
\begin{ans}
$k=v^2/(gR) \approx 0.4$.
\end{ans}
\end{ex}	

\begin{ex} %Сив272 
При выполнении самолетом «мертвой петли», осуществленной впервые русским летчиком П. Н. Нестеровым, сила, действующая на крылья самолета, изменяется по сравнению с их нагрузкой при горизонтальном полете. Пусть самолет Нестерова массой в 3/4 т делает «мертвую петлю» радиуса $R = 235$ м и движется по ней со скоростью 120 км/ч. Найти максимальное значение нагрузки на крылья самолета. Указать, в каком месте траектории эта нагрузка будет максимальной.
\begin{ans}
$F = 14$ кН.
\end{ans}
\end{ex}	

\begin{ex} %Сив276

\begin{figure}[h]
\centering
\includegraphics[width=0.4\textwidth]{CentrafugalMashine3.png}
\caption{}
\label{CentrafugalMashine3}
\end{figure}

Груз массы $M$ может скользить без трения по стержню $a$, укрепленному перпендикулярно к оси вращающейся центробежной машины (рис. \ref{CentrafugalMashine3}). Ось машины вертикальна, и сквозь нее проходит нить, на которой висит груз массы $m$; нить перекинута через блок с и другой ее конец прикреплен к грузу массы $M$. Найти положение груза массы $M$ на стержне $a$ когда центробежная машина вращается с угловой скоростью $\omega$.
\begin{ans}
$R = mg/(M\omega^2)$.
\end{ans}
\end{ex}	

\complexProblems

\begin{ex}  %Сив267

\begin{figure}[h]
\centering
\includegraphics[width=0.2\textwidth]{CentrafugalMashine2.png}
\caption{}
\label{CentrafugalMashine2}
\end{figure}

Шарик радиуса $R$ висит на нити длины $l$ и касается вертикального цилиндра радиуса $r$, установленного на оси центробежной машины (рис. \ref{CentrafugalMashine2}). При какой угловой скорости и вращения центробежной машины шарик перестанет давить на стенку цилиндра?

\begin{ans}
$\omega^2 = g/\sqrt{(R+l)^2-(R+r)^2}$.
\end{ans}
\end{ex}	

\begin{ex} %Сив262

\begin{figure}[h]
\centering
\includegraphics[width=0.5\textwidth]{DeathLoop.png}
\caption{}
\label{DeathLoop}
\end{figure}

Тележка массы $m$ скатывается без трения по изогнутым рельсам, имеющим форму, изображенную на рис. \ref{DeathLoop}. 1) С какой минимальной высоты $h$ должна скатиться тележка для того, чтобы она не покинула рельсов по всей их длине? 2) Какие силы действуют на тележку в наивысшей точке $В$ петли? 3) Каково будет движение тележки, если она скатывается с высоты, меньшей $h$? При решении задачи считать колеса тележки малого размера и малой массы и их вращательного движения не рассматривать.
\begin{ans}
1) $h=5R/2$.
\end{ans}
\end{ex}	

\begin{ex} %Сив283

\begin{figure}[h]
\centering
\includegraphics[width=0.3\textwidth]{CentrafugalMashine.png}
\caption{}
\label{CentrafugalMashine}
\end{figure}

Шарик массы $m$ подвешен на идеальной пружине жесткости $k$ и начальной длины $l$ над центром платформы центробежной машины (рис. \ref{CentrafugalMashine}). Затем шарик начинает вращаться вместе с машиной с угловой скоростью $\omega$. Какой угол $\alpha$ образует при этом пружина с вертикалью?

\begin{ans}
$\alpha = 0$, если $\omega^2 < \frac{kg}{ml_0(g/l_0 + k/m)}$, иначе $\cos \alpha = g\frac{k-m\omega^2}{\omega^2kl_0}$ $\left( \omega < \sqrt{k/m} \right)$.
\end{ans}
\end{ex}	

\clearpage