\begin{Answer}{1}
$v = 22$ м/с.
\end{Answer}
\begin{Answer}{3}
2.
\end{Answer}
\begin{Answer}{4}
Нет.
\end{Answer}
\begin{Answer}{5}
20 м.
\end{Answer}
\begin{Answer}{6}
Разгоняется.
\end{Answer}
\begin{Answer}{7}
$B, E, H, K$
\end{Answer}
\begin{Answer}{8}
1:3:5
\end{Answer}
\begin{Answer}{11}
$s = 12$ м.
\end{Answer}
\begin{Answer}{12}
15 м/c.
\end{Answer}
\begin{Answer}{13}
84 м.
\end{Answer}
\begin{Answer}{14}
2 c.
\end{Answer}
\begin{Answer}{15}
$S = 2 v_0 t$.
\end{Answer}
\begin{Answer}{17}
4 м/с\textsuperscript{2}.
\end{Answer}
\begin{Answer}{19}
$S = vt/\sqrt{2} = 14$ м.
\end{Answer}
\begin{Answer}{20}
20 м/c.
\end{Answer}
\begin{Answer}{21}
10 м; 15 м/c.
\end{Answer}
\begin{Answer}{22}
$s = 150 м$.
\end{Answer}
\begin{Answer}{23}
$t = \sqrt{2D/g}$.
\end{Answer}
\begin{Answer}{24}
$a = \frac{2S\left(t_1-t_2\right)}{t_1t_2\left(t_1+t_2\right)} \approx -3$ м/c\textsuperscript{2} , $v_0 = \frac{S}{t_1}-\frac{at_1}{2} \approx 11,5$ м/с.
\end{Answer}
\begin{Answer}{25}
0,75 с.
\end{Answer}
\begin{Answer}{26}
1) $v_x = v_0 \cos \varphi$, $v_y = v_0 \sin \varphi - gt$, $v = \sqrt{v_{0}^{2} + g^2 t^2 - 2v_{0}gt \sin \varphi}$; 2) $T = \frac{2v_{0}\sin \varphi}{g}$; 3) $ \tg \alpha = \tg \varphi - \frac{gt}{v_0 \cos \varphi}$; 4) $x = v_0 t \cos \varphi$, $y = v_0 t \sin \varphi - \frac{gt^2}{2}$; 5) $y = x \tg \varphi - \frac{gx^2}{2v_{0}^{2} \cos^2 \varphi} $; 6) $H_{\max} = \frac{v_{0}^{2} \sin^2 \varphi}{2g}$; 7) $L_{\max} = v_{0}^2 \sin 2 \varphi / g$; 8) $\varphi^{*} = 45^{\circ}$.
\end{Answer}
\begin{Answer}{27}
$\tg \alpha = v \sqrt{2/hg}$, $l = v \sqrt{2h/g}$.
\end{Answer}
\begin{Answer}{28}
$l = v_0 t \sqrt{2(1-\sin \alpha)} = 22$ м.
\end{Answer}
\begin{Answer}{29}
$v_0 = \sqrt{\frac{lg \cos \alpha}{2 \cos \beta \sin \left( \beta - \alpha \right)}}$.
\end{Answer}
\begin{Answer}{30}
Одновременно.
\end{Answer}
\begin{Answer}{31}
За одинаковое время.
\end{Answer}
\begin{Answer}{32}
Верхняя точка траектории.
\end{Answer}
\begin{Answer}{33}
$v_0 > \sqrt{\frac{g(H^2+L^2)}{2H}}$.
\end{Answer}
\begin{Answer}{34}
$v_0 = 82$ м/с.
\end{Answer}
\begin{Answer}{35}
Нет, $a = 2$ м/с\textsuperscript{2}.
\end{Answer}
\begin{Answer}{36}
$H_1 : H_2 : H_3 = 3 : 2 : 1$; $L_1 : L_2 : L_3 = \sqrt{3} : 2 : \sqrt{3}$.
\end{Answer}
\begin{Answer}{37}
$55,1^{ \circ }$.
\end{Answer}
\begin{Answer}{38}
$L \approx 40$ м, $\varphi \approx 38^{\circ}40^\prime$.
\end{Answer}
\begin{Answer}{39}
17 м/с.
\end{Answer}
\begin{Answer}{40}
$s = \frac{2 v_0^2 \cos \alpha \sin \alpha}{g}$.
\end{Answer}
\begin{Answer}{41}
$l = (v_1 + v_2)\sqrt{v_1 v_2}/g = 2,5$ м.
\end{Answer}
\begin{Answer}{42}
$h = \frac{2u}{g}\left( v \cos \alpha - u \right) \tg^2 \alpha$.
\end{Answer}
\begin{Answer}{43}
$x = v \sqrt{2h/g} \left( 1 + \alpha \right) / \left( 1 - \alpha \right)$.
\end{Answer}
\begin{Answer}{44}
$v = 2 \pi R/T = 3,7$ км/ч; $a = 4\pi^2 R/T^2 = 35$ км/ч\textsuperscript{2}.
\end{Answer}
\begin{Answer}{45}
$\varepsilon = \pi N^2/n$.
\end{Answer}
\begin{Answer}{46}
$a_n = 0.6$ м/с\textsuperscript{2}; $a = 0.67$ м/с\textsuperscript{2}; угол между векторами $\vec{a}$ и $\vec{R}$ составляет $135^{\circ}$.
\end{Answer}
\begin{Answer}{48}
$x= R(\varphi - \sin \varphi) = R(\omega t - \sin \omega t)$, $y=R(1-\cos \varphi) = R(1-\cos \omega t)$, где $\omega = v/R$ -- угловая скорость вращения колеса. Траекторией точек, находящихся на ободе движущегося колеса, будет простая циклоида.
\end{Answer}
\begin{Answer}{57}
A.
\end{Answer}
\begin{Answer}{58}
$90^{\circ}$.
\end{Answer}
\begin{Answer}{61}
$\omega \approx 0.001$ c\textsuperscript{-1}.
\end{Answer}
\begin{Answer}{62}
$v \approx 30$ км/с.
\end{Answer}
\begin{Answer}{63}
$\nu \approx 9$ об/с.
\end{Answer}
\begin{Answer}{64}
$\varepsilon = 4$ рад/c\textsuperscript{2}; $\varphi = 2t^2$ рад.
\end{Answer}
\begin{Answer}{65}
$\varepsilon =  3.14$ рад/\textsuperscript{2}, $N = 25$.
\end{Answer}
\begin{Answer}{66}
5 м/c\textsuperscript{2}, 8.7 м/c\textsuperscript{2}.
\end{Answer}
\begin{Answer}{67}
$a_n = \frac{a_{\tau}^2 t^2}{R} = 2 \varepsilon R \varphi$; $a = \frac{a_{\tau}}{R} \sqrt{R^2 + a_{\tau}^2 t^4} = \varepsilon R \sqrt{1 + 4 \varphi^2}$; $\tg \beta = -\frac{1}{2\varphi}$.
\end{Answer}
\begin{Answer}{68}
$v_x = v_0 (1 + \cos \varphi) = 2v_0 \cos^2 \frac{\varphi}{2}$; $v_y = -v_0 \sin \varphi$; $v = 2v_0 \cos \frac{\varphi}{2}$.
\end{Answer}
\begin{Answer}{69}
$H_{\max} = R + \frac{v^2}{2g} + \frac{gR^2}{2v^2}$; $\cos \varphi^{*} = - \frac{Rg}{v^2}$.
\end{Answer}
\begin{Answer}{70}
1) $\omega = \frac{v}{R \cos \alpha} = 0,6$ рад/с; 2) $\varepsilon = (v/R)^2 \tg \alpha = 2,3$ рад/c\textsuperscript{2}.
\end{Answer}
\begin{Answer}{71}
1) 5 Н; 2) 15 Н.
\end{Answer}
\begin{Answer}{72}
$a = \frac{mg}{m+M}$; $T = \frac{mMg}{m+M}$.
\end{Answer}
\begin{Answer}{73}
$a = g (\sin \alpha - \mu \cos \alpha)$, при $\tg \alpha > \mu$.
\end{Answer}
\begin{Answer}{74}
$a = \frac{m_1 \sin \alpha - m_2}{m_1 + m_2}g$; $T = \frac{m_1 m_2}{m_1 + m_2}\left( 1+ \sin \alpha \right)g$.
\end{Answer}
\begin{Answer}{75}
$a = \frac{M(m_1 + m_2) - 4 m_1 m_2 \sin \alpha}{M(m_1 + m_2) + 4 m_1 m_2}g$.
\end{Answer}
\begin{Answer}{77}
$F$.
\end{Answer}
\begin{Answer}{82}
$a = \frac{Mg}{M + m_1 +m_2 +m_3}$, $T_1 = (m_1 +m_2 +m_3)a$, $T_2 = (m_2 +m_3)a$, $T_3 = m_3 a$.
\end{Answer}
\begin{Answer}{83}
200 H.
\end{Answer}
\begin{Answer}{85}
На лошадь со стороны Земли действует сила $F = M(\mu g + a) + ma = 1,17$ кН.
\end{Answer}
\begin{Answer}{86}
$m = 2(M - P/g)$.
\end{Answer}
\begin{Answer}{87}
$ v = \sqrt{2gh(1 - \mu \cot \alpha)} = 9 \textrm{ м/c}.$
\end{Answer}
\begin{Answer}{88}
$a = (m_1 - m_2)g / (m_1 + m_2)$, $T = 2 m_1 m_2 g /(m_1 + m_2)$, $F = 2T$.
\end{Answer}
\begin{Answer}{89}
$a_1 = (2m_1-m_2)g/(2m_1+0,5m_2)$, $a_2 = -a_1/2$, $T=3m_1 m_2g/(4m_1 + m_2)$.
\end{Answer}
\begin{Answer}{91}
$a_1 = \frac{m_1(m_2+m_3)-4m_2m_3}{m_1(m_2+m_3)+4m_2m_3}g$, $T= \frac{8m_1m_2m_3g}{m_1(m_2+m_3)+4m_2m_3}$, $T_2 = T_1/2$.
\end{Answer}
\begin{Answer}{92}
$\mu = \tg \alpha (k^2-1)/(k^2+1) = 0,16$.
\end{Answer}
\begin{Answer}{93}
$v = \frac{mg}{r}\left[ \left( \frac{v_0r}{mg} +1 \right)e^{-\frac{rt}{m}} - 1 \right]$.
\end{Answer}
\begin{Answer}{94}
$v=10$ см/с.
\end{Answer}
\begin{Answer}{95}
$F = \frac{m^2 v^2}{2SM}$
\end{Answer}
\begin{Answer}{96}
$v_1 = \frac{m_1(v+u)+mv}{m+m_1}$, $v_2 = v$, $v_3 = \frac{m_1(v-u)+mv}{m+m_1}$.
\end{Answer}
\begin{Answer}{97}
$v = \frac{m_1v_1 + m_2v_2}{m_1+m_2}$, $K = \frac{(m_1v_1+m_2v_2)^2}{2(m_1+m_2)}$.
\end{Answer}
\begin{Answer}{98}
$S_2 = 5$ км.
\end{Answer}
\begin{Answer}{105}
1 км/c.
\end{Answer}
\begin{Answer}{106}
1.4 кгм/c.
\end{Answer}
\begin{Answer}{107}
1) 6.3 м/с; 2) 0.57 м/с.
\end{Answer}
\begin{Answer}{108}
$F = 2Mv/\tau$.
\end{Answer}
\begin{Answer}{109}
$F = \Delta m v/ \Delta t$.
\end{Answer}
\begin{Answer}{110}
9 м/с и 1 м/с.
\end{Answer}
\begin{Answer}{111}
$\vec{w} = \frac{M\vec{u}}{M+m}$, $\vec{v} = -\frac{m\vec{u}}{M+m}$.
\end{Answer}
\begin{Answer}{112}
$\vec{S_1} = -\frac{m\vec{l}}{M+m}$, $\vec{S_2} = \frac{M\vec{l}}{M+m}$.
\end{Answer}
\begin{Answer}{113}
$u = \sqrt{9v_{1}^2 + 2v^2} = 14$ м/с.
\end{Answer}
\begin{Answer}{114}
$p = (2m/3)\sqrt{2gl} = 3,5$ кг$\cdot$м/c.
\end{Answer}
\begin{Answer}{115}
$F_1 = Q\sqrt{2gh} \approx 140$ Н, $F_2 = \sqrt{2} F_1 \approx 200$ Н, $F_3 = F_1/\sqrt{2} \approx 100$ Н.
\end{Answer}
\begin{Answer}{116}
4,25 м; $\approx$ 8,16 м/с.
\end{Answer}
\begin{Answer}{117}
4,3 МДж.
\end{Answer}
\begin{Answer}{118}
$N =\frac{mv_{0}^3}{4l}$.
\end{Answer}
\begin{Answer}{119}
После соударения второй шарик отскочит назад, $A = \frac{2m_2v}{m_1+m_2}\sqrt{\frac{m_1}{k}}$.
\end{Answer}
\begin{Answer}{120}
$v = 2\frac{M+m}{m}\sqrt{lg}\sin \frac{\alpha}{2}$.
\end{Answer}
\begin{Answer}{133}
$A=mg(H+\mu L)$.
\end{Answer}
\begin{Answer}{134}
$A \approx 10$ Дж.
\end{Answer}
\begin{Answer}{135}
$N = mgv(\mu \cos \alpha +\sin \alpha)$.
\end{Answer}
\begin{Answer}{136}
$S = \frac{v^2}{4 \mu g}$.
\end{Answer}
\begin{Answer}{137}
$v_1 = 2\sqrt{gl}$, $v_2 = \sqrt{5gl}$.
\end{Answer}
\begin{Answer}{138}
1) $U_1/U_2 = k_2/k_1$; 2) $U_1/U_2 = k_1/k_2$.
\end{Answer}
\begin{Answer}{139}
1) $K=\frac{m^2v^2}{2(M+m)}$; 2) $E_1 = \frac{Mmv^2}{2(M+m)}$; 3) $A = v\sqrt{\frac{Mm}{k(M+m)}}$.
\end{Answer}
\begin{Answer}{140}
$x = Ml/(M-m)$.
\end{Answer}
\begin{Answer}{141}
$\alpha_1 =\frac{m_1-m_2}{m_1+m_2}\alpha$, $\alpha_2 =\frac{2m_1}{m_1+m_2}\alpha \sqrt{\frac{l_1}{l_2}}$.
\end{Answer}
\begin{Answer}{142}
$\alpha = \arccos(2/3)$.
\end{Answer}
\begin{Answer}{143}
$v_0 = \sqrt{gR} \approx 8$ км/с.
\end{Answer}
\begin{Answer}{144}
$F_1 \approx 5,5$ кН; $F_2 \approx 4$ кН.
\end{Answer}
\begin{Answer}{145}
$\omega^2 = \frac{g(\cos \alpha - \mu \sin \alpha)}{h \tg \alpha (\sin \alpha + \mu \cos \alpha)}$.
\end{Answer}
\begin{Answer}{146}
$v = \sqrt{Rg \ctg \alpha}$.
\end{Answer}
\begin{Answer}{147}
$x = (g/\omega^2) \tg \alpha$.
\end{Answer}
\begin{Answer}{154}
Затормозить.
\end{Answer}
\begin{Answer}{155}
1) $F=mg$; 2) $F=mg-\frac{mv^2}{R}$; 3) $F=mg+\frac{mv^2}{R}$.
\end{Answer}
\begin{Answer}{156}
$v_0 = \sqrt{gR}$.
\end{Answer}
\begin{Answer}{157}
$k=v^2/(gR) \approx 0.4$.
\end{Answer}
\begin{Answer}{158}
$F = 14$ кН.
\end{Answer}
\begin{Answer}{159}
$R = mg/(M\omega^2)$.
\end{Answer}
\begin{Answer}{160}
$\omega^2 = g/\sqrt{(R+l)^2-(R+r)^2}$.
\end{Answer}
\begin{Answer}{161}
1) $h=5R/2$.
\end{Answer}
\begin{Answer}{162}
$\alpha = 0$, если $\omega^2 < \frac{kg}{ml_0(g/l_0 + k/m)}$, иначе $\cos \alpha = g\frac{k-m\omega^2}{\omega^2kl_0}$ $\left( \omega < \sqrt{k/m} \right)$.
\end{Answer}
\begin{Answer}{163}
$a= \frac{m_2-m_1}{m_1+m_2+I/r^2}g$, $T_1 = \frac{2m_1m_2g + m_1gI/r^2}{m_1+m_2+I/r^2}$, $T_2 = \frac{2m_1m_2g + m_2gI/r^2}{m_1+m_2+I/r^2}$.
\end{Answer}
\begin{Answer}{164}
$\varphi = \frac{gt^2}{2R(1+Mg/2P)}$.
\end{Answer}
\begin{Answer}{165}
$a = \frac{2(M+m)r^2}{mr^2+MR^2+2(M+m)r^2}g$.
\end{Answer}
\begin{Answer}{166}
$a=2/3g\sin \alpha$.
\end{Answer}
\begin{Answer}{167}
$a = \frac{F(R\cos \alpha - r)}{I + mR^2}$.
\end{Answer}
\begin{Answer}{178}
$\varepsilon = \frac{m_2R - m_1r}{m_2R^2 + m_1r^2 + I}g$, $T_1 = m_1(g+r\varepsilon)$, $T_2 = m_2(g-R\varepsilon)$.
\end{Answer}
\begin{Answer}{179}
$\Delta m = m/(1+I/mr^2)$.
\end{Answer}
\begin{Answer}{180}
$a = \frac{M+2m}{4m+M+I/r^2}g$, $T_1 = (2m+I/r^2)a - mg$, $T_2=m(g-2a)$.
\end{Answer}
\begin{Answer}{181}
$a=5/7g\sin \alpha$.
\end{Answer}
\begin{Answer}{182}
$F = 1/3 mg \sin \alpha$.
\end{Answer}
\begin{Answer}{183}
$\mu > 1/3 \tg \alpha$.
\end{Answer}
\begin{Answer}{184}
$a = m_3g/(m_1 + 6m_2 +m_3)$.
\end{Answer}
\begin{Answer}{185}
$\omega = \sqrt{3g/l}$.
\end{Answer}
\begin{Answer}{186}
Скорость вращения возрастет в $(1+mR^2/I)$ раз.
\end{Answer}
\begin{Answer}{187}
$\omega = mrv/(0,5MR^2 + mr^2)$.
\end{Answer}
\begin{Answer}{188}
$l = L/\sqrt{3}$.
\end{Answer}
\begin{Answer}{189}
Лозу следует рубить участком сабли, отстоящим на $2/3$ длины от ручки сабли.
\end{Answer}
\begin{Answer}{193}
$\omega = mvr/(I+mr^2) = 1$ рад/с.
\end{Answer}
\begin{Answer}{194}
$M = E / (2 \pi N) = 2$ Нм.
\end{Answer}
\begin{Answer}{196}
1) $\Delta E = 2I_1^2\omega^2/I_2$; 2) $\Delta E = I_1^2\omega^2/2I_2$.
\end{Answer}
\begin{Answer}{197}
$v = v_0 -M/m \sqrt{2gL/3} \sin \alpha /2 = 444$ м/с.
\end{Answer}
\begin{Answer}{198}
$t = m\omega r^2 / 2M_0$, $N = M_0t^2/2I$, $I = I_0 + m(d^2 + r^2/2)$.
\end{Answer}
\begin{Answer}{199}
$\Delta E = I_1 I_2 (\omega_1 - \omega_2)^2/ 2(I_1 + I_2)$.
\end{Answer}
\begin{Answer}{200}
$ML^2 = ml^2$, при $M \leq m$.
\end{Answer}
\begin{Answer}{201}
$\omega_1 = \frac{M_1r_1^2\omega_0}{M_1r_1^2 + M_2a^2}$, $\omega_2 = \frac{M_1r_1^2 a \omega_0}{(M_1r_1^2 + M_2a^2)r_2}$, $\Delta E = \frac{M_1 M_2r_1^2a^2\omega_0^2}{4(M_1r_1^2 + M_2a^2)}$.
\end{Answer}
\begin{Answer}{202}
После удара шарик и стержень будут подниматься как единое тело на высоту $h = \frac{6m^2}{(M+2m)(M+3m)}H$.
\end{Answer}
\begin{Answer}{203}
$N = 2\pi r n^2/(4 \mu g)$.
\end{Answer}
\begin{Answer}{204}
$v = 2M\sqrt{3gl}/(M+3m)$.
\end{Answer}
\begin{Answer}{205}
$a = g/3$, при $\mu \leq 2/9$.
\end{Answer}
\begin{Answer}{206}
$v = 2/7r \omega_0$, $\Delta E = 1/7 mr^2 \omega_0^2$.
\end{Answer}
\begin{Answer}{209}
$v = \sqrt{3gl}$, $x = 2/3l$.
\end{Answer}
\begin{Answer}{210}
Если при оценке человека моделировать однородным стержнем, получим $F = 4mg$.
\end{Answer}
\begin{Answer}{211}
$T = Mg/(1 + Mr^2/(4mR^2))$, $T_1 = Mg + MhrT/(\pi m R^2)$.
\end{Answer}
\begin{Answer}{212}
$a = g/(1+3m/8M)$, при отсутствии скольжения $\mu /leq (8+3m/M)^{-1}$; $a = g(1 - \mu m/3M)/(1+m/3M)$, при скольжении $\mu < (8+3m/M)^{-1}$.
\end{Answer}
\begin{Answer}{213}
$\varepsilon = \frac{2Rg \sin \alpha}{(I_1 + I_2)/m + 2R^2} \approx 66$ c\textsuperscript{-2}.
\end{Answer}
\begin{Answer}{214}
Движение после перехода границы будет сначала равнозамедленное, а затем с постоянной скоростью; 1/3 энергии превратится в тепло, 2/9 — во вращательную энергию и 4/9 останется в виде энергии поступательного движения.
\end{Answer}
\begin{Answer}{215}
$\cos \alpha = 10/17$.
\end{Answer}
\begin{Answer}{216}
$v_m = \omega A$, $a_m = \omega^2 A$.
\end{Answer}
\begin{Answer}{217}
$T = 4 \sqrt{\pi m / g \rho D^2}$.
\end{Answer}
\begin{Answer}{218}
$T = \pi \sqrt{m L /P}$.
\end{Answer}
\begin{Answer}{219}
При $\omega^2 A > g$ грузик будет подскакивать.
\end{Answer}
\begin{Answer}{220}
$A = \frac{mg}{k}\sqrt{ 1 + \frac{2hk}{mg}}$.
\end{Answer}
\begin{Answer}{236}
$A \approx 6,2$ см.
\end{Answer}
\begin{Answer}{237}
$E_1 = m A^2 \omega^2 (1 - \cos 2 \omega t)/4$, $E_2 = m A^2 \omega^2 (1 + \cos 2 \omega t)/4$, $E = m A^2 \omega^2 /2$.
\end{Answer}
\begin{Answer}{238}
$\mu = 4 \pi^2 A / (gT^2) = 0,1$.
\end{Answer}
\begin{Answer}{239}
Гармоническое колебание с периодом $T = 2 \pi \sqrt{R_0 / g_0}$, где $R_0$ -- радиус земного шара, $g_0$ -- ускорение свободного падения на поверхности Земли.
\end{Answer}
\begin{Answer}{240}
$x = mg/k(1-\cos \sqrt{g/m} t)$.
\end{Answer}
\begin{Answer}{241}
$x = l/(2\sqrt{3})$, $\omega^2 = g\sqrt{3} / l$.
\end{Answer}
\begin{Answer}{242}
$T =  \pi\sqrt{3m/k}$.
\end{Answer}
\begin{Answer}{243}
$T = 2 \pi \sqrt{I/(mga + ka^2)}$.
\end{Answer}
\begin{Answer}{244}
$x_0 = mga/kl$, $T = 2 \pi \sqrt{I/kl^2}$.
\end{Answer}
\begin{Answer}{248}
$l = 15$ см.
\end{Answer}
\begin{Answer}{249}
$T = 2 \pi (M+m) / mv = 1,26$ с.
\end{Answer}
\begin{Answer}{250}
$T = 2 \pi \sqrt{(I+mx^2)/(Ma - mx)}$.
\end{Answer}
\begin{Answer}{251}
$T = 2 \pi \sqrt{m(1/k_1 + 1/k_2)}$.
\end{Answer}
\begin{Answer}{252}
$T = 2 \pi \sqrt{m_1m_2/(m_1 + m_2)k}$.
\end{Answer}
\begin{Answer}{253}
$T = 2 \pi \sqrt{L/3g}$.
\end{Answer}
